\documentclass[../main.tex]{subfiles}

\begin{document}
\section{Discussion}

\subsection{Validation}

% Validation was good
The behavioral experiment showed that GANalyze was indeed able to generate images of increasing aesthetic value. Even when forced to choose between two nearly identical images, our participants picked the image GANalyze generated to be slightly more beautiful at rates above chance. The relation between GANalyze's $\alpha$-values and our participant's preference followed the shape of a typical psychometric curve in 2-AFC discrimination tasks. This implies the internal values of the neural network translate to human aesthetic appreciation similar to how for example auditory stimulus intensity translates to human auditory perception. This relation hints towards objective properties of beauty.
%Good psychophysical relation between $\alpha$-value and rating. Positive slope shows that participants are in agreement with GANalyze, therefore we argue that using GANalyze to study the aesthetic properties of images is justified.

%Steep curve represents good stimulus representation


%\subsubsection{Visual Definitions}
%Images here

%\subsection{GANalyze Parameters}
%st2

% Interesting follow-up question: Would turning up parameters like brightness, saturation etc result in the same effect? Or has the GAN learned a specific way of applying these parameters?

\end{document}