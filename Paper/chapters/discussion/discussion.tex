\documentclass[../main.tex]{subfiles}

\begin{document}
\section{Discussion}

\subsection{Validation}
% Validation was good
The behavioral experiment showed that GANalyze was indeed able to generate images of increasing aesthetic value, as seen in the positive slope between the proportion agreement with the neural network and the $\alpha$-values. Even when forced to choose between two nearly identical images, our participants picked the image GANalyze generated to be more beautiful at rates above chance. The relation between GANalyze's $\alpha$-values and our participant's preference followed the shape of a typical psychometric curve in 2-AFC discrimination tasks, implying that the internal values of the neural network translate to human aesthetic appreciation similar to how for example auditory stimulus intensity translates to human auditory perception.  Taken together, these findings hint towards objective properties of beauty. As we have confirmed that GANalyze is indeed able to manipulate the aesthetic qualities of images, we can now use these images to find what makes an image beautiful.

\subsection{Behavioral Findings}
We found that while our participants were more likely to agree with the network for higher $\alpha$-values, their responses to high $\alpha$-values was different from low $\alpha$-values.
%Steep curve represents good stimulus representation

% Cross-cultural effects etc

%\subsubsection{Visual Definitions}
%Images here

%\subsection{GANalyze Parameters}
%st2

% Interesting follow-up question: Would turning up parameters like brightness, saturation etc result in the same effect? Or has the GAN learned a specific way of applying these parameters?

\end{document}