\documentclass[../main.tex]{subfiles}

\begin{document}
\newpage
\section{Abstract}
Finding the properties underlying beauty has always been a prominent yet difficult topic to approach, particularly from a scientific framework. Even so, throughout the history of science, new technological developments have often aided scientific progress by expanding the scientists' toolkit. Currently in the spotlight of cognitive psychology are deep neural networks. In this study, we have used a Generative Adversarial Network (GAN) to generate images of increasing aesthetic value. We validated that this network indeed was able to increase the aesthetic value of an image by letting participants decide which of two presented images they considered to be more beautiful. As our validation was successful, we were justified to use the generated images to extract low-level features contributing to their aesthetic value. We compared the brightness, contrast, sharpness, and saturation levels of `low-aesthetic' images to those of `high-aesthetic' images. We found that all of these features increased for the beautiful images, implying that they may play an important role underlying the aesthetic value of an image. With this study, we have provided further evidence for the potential value GANs may have for research concerning beauty.

\end{document}