\documentclass[../main.tex]{subfiles}

\begin{document}
\newpage
\section{Abstract}
The search for a definition of beauty has always been a popular yet difficult topic to approach. Throughout the history of science, new technological developments have often led to vast scientific advances. Currently in the spotlight of cognitive psychology are deep neural networks. In this study, we have used a Generative Adversarial Network (GAN) to generate images of increasing aesthetic value. We validated that this network indeed had an understanding of what it means for an image to be beautiful by letting participants decide which of two presented images they considered to be more beautiful. As the network was validated to indeed make images more beautiful, we were justified to use the generated images to extract low-level features contributing to their aesthetic value. We compared the brightness, contrast, sharpness, and saturation levels of `low-aesthetic' images to those of `high-aesthetic' images. We found that all of these features increase for the beautiful images, implying that they may play an important role underlying the aesthetic value of an image. With this study, we have shown that GANs can potentially become very influential in research concerning beauty.

\end{document}