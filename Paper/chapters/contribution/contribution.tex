\documentclass[../main.tex]{subfiles}

\begin{document}
\newpage
\section{Personal Contribution}
After getting assigned this thesis topic, I had a meeting with my supervisor, professor Johan Wagemans, to discuss the general ideas and methods to be used in the project. I was given the options to either work through this project while being guided for every step, or to work primarily independently and discuss only the major steps. I chose to work primarily independently.

As this thesis was largely based on GANalyze, a model previously developed by Lore Goetschalckx, Alex Andonian, Aude Oliva, and Phillip Isola, I had a solid foundation to build on. To adapt GANalyze to aesthetics, I cloned their public repository containing the full model and worked with their Python code to obtain the desired output. As I ran into trouble here, my day-to-day advisor, Anne-Sofie Maerten, helped me figuring out the issues I had, eventually leading to a successfully trained GAN.

I proposed my ideas for the behavioral task and its details in a meeting with my supervisor. Together we made some minor adjustments which resulted in the experiment used in this thesis. Because I required a large number of participants, I decided to make an online experiment hosted on Prolific.co. Experiments hosted in web browsers are typically made in JavaScript which I had no prior experience with. Regardless, I created most of the experiment myself with some assistance from Anne-Sofie and Christophe Bossens. After making the study publicly available on Prolific.co, we got data from 650 participants thanks to funding provided by professor Wagemans. Unfortunately, the first 100 participants got a wrong version of the experiment, making their data unusable for analysis. Anne-Sofie also helped me reject participants providing invalid data. After acquiring data I did all of the analysis myself in R. Methods used include structural equation modeling, analysis of variance, fitting psychometric functions, polynomial regression, and plotting figures. To extract image features from the output, I made a Python project from scratch that was able to extract all features I was interested in. 

After having acquired and analyzed most of the data, I started writing the paper. I first sent a preliminary draft to Anne-Sofie who provided me with useful feedback that I immediately implemented. Later, I sent an improved draft to professor Wagemans who also provided me with good feedback, helping me put the finishing touches on this thesis. Despite me choosing to work primarily independently, this all would have been impossible without all the help I received.

In the spirit of open science, I made this project (excluding participants' demographic information to respect their privacy) public on GitHub, accessible at: \newline \url{https://github.com/amar-music/beautification-of-images}.

\end{document}