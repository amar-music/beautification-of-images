\documentclass[../main.tex]{subfiles}

\begin{document}
\section{Appendix}
\subsection{Appendix A: Shortened Aesthetic Experience Questionnaire} 
	
	\textbf{In general, when I view art...}
		
	1. I experience a wide range of emotions. (emotional)
	
	2. I compare the past culture of the art with present-day culture. (cultural)
	
	3. The composition of a work of art is important to me. (perceptual)
	
	
	4. I try to understand the work completely. (understanding)
	
	
	5. I have a clear idea of what to look for when viewing the work of art. (flow-proximal)
	
	
	6. I lose track of time when I view the work of art. (flow-experience)

\newpage

\subsection{Appendix B: Task Instructions}

\begin{flushleft}

For this study, a Generative Adversarial Network (GAN) was trained to make images more beautiful. The goal of this study is to test whether this network really has some "understanding" of aesthetics. \newline

Before the task, you will be asked to indicate how much experience you have with art. In addition, demographic information linked to your Prolific account (such as age, sex, nationality, etc.) will be linked to your responses.

During the task, you will be presented with two similar looking images produced by the GAN. The images are assumed to be of differing aesthetic quality (according to the GAN). Your task is to indicate which of the two images you believe is more aesthetic. You indicate this with the key on your keyboard corresponding to the location of your preferred image. If you prefer the left image, you press F on your keyboard and if you prefer the right image, you press J. \newline

Because this study aims to test the GANs understanding of aesthetics, it is important that you pick the image that you really feel is better looking and not what you expect the GAN to produce. The task will be 10 minutes with a short break after 5 minutes. You should not take too much time during each trial, and it is recommended to pick whichever image you prefer at first glance without analyzing too many details. \newline

If you press at random during the whole experiment or provide irrational data in another way, your submission will be rejected and you will not get paid. \newline

This study has been approved by the Social and Societal Ethics Committee of the KU Leuven (SMEC). \newline

In case of complaints or other concerns with regard to the ethical aspects of this research you can contact smec@kuleuven.be. \newline

If you continue to the experiment you will have given consent for participation and data collection.
\end{flushleft}

\end{document}