\documentclass[../main.tex]{subfiles}

\begin{document}
\section{Conclusion}
% Restate research question/hypothesis and restate major findings
The goal of this study was to find out what makes an image beautiful and whether we can use a GAN as a means to answer this question. To do so, we trained GANalyze to generate images of increasing aesthetic value. Using these images, we validated the GAN's understanding of aesthetics, allowing us to utilize these images to find features that underlie the aesthetic value an image. With these images, we constructed a so-called visual definition of beauty. By extracting low-level features from these images, we found that brightness, contrast, sharpness, and saturation were properties that consistently made images more beautiful.

% Major contributions to existing literature
In a broader sense, we showed that GANs have a potential future in studies on aesthetic appreciation. A significant benefit of GANs is that they can be used to generate rich stimuli that are easy to manipulate seem to be, as we have shown in this study, ecologically valid. In a philosophical sense, we provided additional evidence for objective properties of beauty observed across cultures.

% Limitations
A limitation of our study is that it was restricted to comparisons of two very similar images. This shortcoming can be addressed by conducting a new study using our existing dataset to compare different image sequences with similar aesthetic ratings. In addition, our study did not delve quite deep enough into the subjective features of aesthetic appraisal.

% Future directions
GANs and artificial neural networks in general are proving to be very useful in conducting research on aesthetics and psychology as a whole. These powerful new methods may prove to be a crucial tool leading to groundbreaking discoveries in cognitive psychology.


\end{document}